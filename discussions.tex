\section{Discussion}
\label{sec:discussion}
% What did we achieve?
The design of the actuators and fabrication methods described in this paper provide recipes for the rapid fabrication of modular soft robots with arbitrary body morphology.
%
%New capabilities
This class of soft robots is very well-suited for tasks requiring: (i) interactions with humans and environments to be safe, (ii) uncertainty to be mitigated at the hardware level, (iii) continuous and dexterous deformation, and/or (iv) hardware to take an unstructured, amorphous form.
%
For example, by making robots from soft elastic materials, with no sharp edges, and with relatively low link inertia, a robot's reliance on sensors and software for safety is reduced.
%
The prospects for safe integrations between a robot and human are generally increased when the compliance of the material composing the machine match those of soft biological materials \citet{majidi2014soft}, and this feature is inherent to robots made of soft silicone elastomer.
%
An alternative approach is to allow resilient soft machines to handle some uncertainty at the hardware level in order to reduce the burden on the computational system.
%
For example, consider how a soft manipulator passively conforms to the environment's boundary. The planner is unaware of this complex interaction, but the primary task can still be successfully executed.
%
Further, modern inspection tasks as well as invasive surgery require devices with redundant degrees of freedom and high dexterity and often impose the constraint of navigating around sensitive objects.
%
As initially demonstrated here, soft robotic manipulators may be well-suited for this class of tasks.

%What are the limitations and open problems



%Lastly, this work brings to attention the need for proprioceptive localization systems for soft robots. Feedback for our controller comes currently from an exteroceptive localization system. This is a reasonable method for indoor, laboratory or factory environments where there is sufficient line of sight to the manipulator. However, this sensor approach is prohibitive in that the environment must be equipped with cameras and the tasks cannot occlude the view of these cameras onto the manipulator. Proprioceptive localization for soft robots is an important avenue for future research.
