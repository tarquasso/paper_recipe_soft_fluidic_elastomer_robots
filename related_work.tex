\section{Related Work}
\label{sec:Related Work}

\subsection{Actuation}
\label{subsec:Related Work, Actuation}
So far, the design of existing so-called "soft" manipulators, which are position-controlled and have multiple DOFs, are actually not soft.
Originally, many rigid hyper-redundant and rigid continuum robots \citep{hannan2003kinematics} \citep{cieslak1999elephant} \citep{buckingham2002snake} used an array of servomotors or linear actuators to pull cables that moved rigid connecting plates located between body segments.
Some soft robots have adopted a similar actuation scheme consisting of tendons pulling rigid fixtures embedded on a continuously deformable backbone as seen in the soft manipulators controlled by \citet{gravagne2002uniform}, \citet{mcmahan2005design}, and \citet{camarillo2009configuration}.
There is an example of a position-controlled soft rubber arm using cables without rigid plates developed by \citet{wang2013visual}, but the arm consists of only one actuated segment and therefore does not require internal fixtures.
Another common design of position controlled soft manipulators involves distributed pneumatic muscle actuators (PMAs).
Here, PMAs are embedded throughout the robot's body.
Notable examples include the manipulators developed by \citet{mcmahan2006field}, \citet{pritts2004design}, and \citet{kang2013design}.
\citet{mcmahan2006field} uses 18 air muscle actuators distributed throughout four arm segments.
\citet{pritts2004design} uses 14 McKibben actuators within two body segments.
\citet{kang2013design} uses 24 PMAs within 6 body segments.
Again, these designs are not entirely soft, because rigid plates are included between the segments for actuator mounting and as kinematic constraints.

%\subsection{Control}
%\label{subsec:Related Work, Control}
%To the best of our knowledge, highly compliant robots, whose bodies are made from soft rubber, and distributed pneumatic actuators are not capable of closed-loop curvature control.
%Prior works in this field use open-loop control, but this approach is not sufficient for providing accurate control of body segment curvature during the execution of novel tasks.
%Most fluid-powered soft robots use open-loop valve sequencing to control body segment bending.
%Valve sequencing means that a valve is turned on for a duration of time to pressurize the actuator and then turned off to either hold or deflate it.
%For instance, there are soft rolling robots developed by \citet{correll2010soft}, \citet{onal2011soft}, and \citet{marchese2011soft} made of Fluidic Elastomer Actuators (FEAs) that use this control approach.
%Also a soft snake-like robot developed by \citet{onal2013autonomous} uses this open-loop scheme to control eight distributed FEAs among four body segments to enable serpentine locomotion.
%Again, \citet{shepherd2011multigait} use an open-loop valve controller to drive body segment bending in an entirely soft multi-gait robot and then passive control in an explosive, jumping robot \citep{shepherd2013using}.
%\citet{martinez2013robotic} develop manually operated elastomer tentacles containing nine PneuNet actuators embedded within three body segments.
%There is also an example of controlling a soft pneumatic inchworm-like robot using servo-controlled pressure described in \citet{lianzhi2010}.
%Here, a PWM approach is used to drive rapid valve switching to continuously vary airflow.
%
%Open-loop control is also common for soft rubber robots that do not use pneumatic actuation.
%For example, previous work on soft bio-inspired octopus-like arms developed by \citet{calisti2010study} demonstrate open-loop capabilities like grasping and locomotion \citep{laschi2012soft, calisti2011octopus}.
%\citet{umedachi2013highly} developed a soft crawling robot that uses an open-loop SMA driver to control body bending.
%
%\subsection{Kinematics}
%\label{subsec:Related Work, Kinematics}
%Despite variability in the design of soft continuum robots \citep{gravagne2002uniform, pritts2004design, mcmahan2005design, mcmahan2006field, chen2006development, camarillo2009configuration, kang2013design, wang2013visual}, their kinematics are often represented using a piecewise constant curvature model.
%The piecewise constant curvature assumption means each body segment of a multi-segment arm is assumed to deform with constant curvature.
%This representation for continuum robots is reviewed by \citet{webster2010design}.
%\citet{hannan2003kinematics} provide one of the first examples of the piecewise constant curvature model.
%As Webster's review discusses, the generality of this modeling assumption is due to the physics behind the deformation. Specifically, \citet{gravagne2003large} and \citet{li2002design} show a moment applied by a guided cable fixed to the end of a continuum backbone produces constant curvature along the backbone.
%\citet{jones2006practical} show that the constant curvature concept also applies to pneumatic muscle actuators bending a continuum backbone.
%Recently, \citet{onal2011soft} show that rectangular fluidic elastomer actuators with serpentine channels deform along an arc of constant curvature.

%\subsection{Planning}
%\label{subsec:Related Work, Planning}
%A limitation of existing approaches in solving the inverse kinematics problem for soft continuum arms is that the whole arm, in addition to the end effector's pose, is not considered in the solution.
%Autonomous obstacle avoidance and movement through a confined environment is difficult without a computational solution for the inverse kinematics problem that is aware of the robot's whole arm in space.
%\citet{buckingham2002snake} articulates as a distinguishing advantage of a snake-like arm, that it can potentially achieve the primary task of tip control, while meeting the secondary task of shaping the whole arm.
%\citet{neppalli2009closed} provide a closed-form inverse kinematics solution for continuum arms, but the Jacobian-based solution only considers the endpoint of the final body segment and obstacle avoidance requires manual planning.
%\citet{jones2006kinematics} control Air-OCTOR and OctArm using real-time Jacobian-based control over task-space, but rely on joystick control for whole arm tasks like manipulation and grasping \citep{csencsits2005user}.
%Local optimization has shown promising results for rigid-bodied redundant manipulators \citep{nenchev1989redundancy}, but as far as we are aware of such a technique has not been used to solve the whole body manipulation problem for a soft robot.
%Furthermore, we are not aware of any existing soft-bodied fluidic robots with highly deformable exterior envelopes \citep{correll2010soft, onal2011soft, onal2013autonomous, marchese2011soft, marchese2014design, shepherd2011multigait, shepherd2013using, martinez2013robotic} that consider whole body manipulation when moving in task-space.
%With fewer kinematic constraints, the envelop of these soft robots expand or radially bulge at locations along the body under actuation.
%Accordingly, whole body planning for soft and highly compliant robots must take into consideration this dynamic envelope.
%
%\subsection{Grasping}
%\label{subsec:Related Work, Grasping}
%There are several examples of soft fluidic grippers described in recent literature.
%\citet{deimel2013compliant} developed a pneumatically actuated three-fingered hand made of reinforced silicone that is mounted to a hard robot and capable of robust grasping.
%More recently, they have developed an anthropomorphic soft pneumatic hand capable of dexterous grasps \citep{deimel2014novel}.
%\citet{ilievski2011soft} create a pneumatic starfish-like gripper composed of silicone membranes and demonstrate how it can grasp an egg.
%\citet{Stokes2014hybrid} use a soft elastomer quadrupedal robot to grasp objects in a hard-soft hybrid robotic platform.
%A puncture resistant soft pneumatic gripper is developed in \citet{shepherd2013soft}.
%An alternative to positive pressure actuated soft grippers is the robotic gripper based on the jamming of granular material developed in \citet{brown2010universal}.
%Perhaps the most related soft pneumatic actuator design to our current work is the Pneu-net designs by \citet{mosadegh2014pneumatic} and by \citet{polygerinos2013towards}.
%These finger-like actuators can deform with minimal volume change and leverage a pleated channel morphology.

\subsection{Soft-bodied Robots}
\label{subsec:Related Work, Soft Robots}
More generally, the literature has several examples of soft pneumatic elastomer robots.
Many of these robots use open-loop controllers.
There are soft fluid-powered rolling robots \citep{correll2010soft, onal2011soft, marchese2011soft}.
Also, there are soft-bodied robotic fish powered by pneumatics \citep{marchese2014autonomous} and by hydraulics \citep{katzschmann2014hydraulic}.
There are soft-legged pneumatic walkers \citep{shepherd2011multigait, tolley2014resilient}, explosive jumpers \citep{shepherd2013using}, and tentacles \citep{martinez2013robotic}.


